\newacronym{ADC}{ADC}{Analog to Digital Converter}
\newacronym{DAC}{DAC}{Digital to Analog Converter}
\newacronym{LED}{LED}{Light Emission Diode}
\newacronym{KB}{KB}{Kilobyte}
\newacronym{RAM}{RAM}{Random Access Memory}
\newacronym{ROM}{ROM}{Read-Only Memory}
\newacronym{DSP}{DSP}{Digital Signal Processor (Processador Digital de Senyal)}
\newacronym{FPGA}{FPGA}{Field Programmable Gate Array}
\newacronym{ASIC}{ASIC}{Application Specific Integrated Circtuit (circuit integrat d'aplicació específica)}
\newacronym{Silabs}{SiLabs}{Silicon Labs}
\newacronym{TI}{TI}{Texas Instruments}
\newacronym{NXP}{NXP}{NXP Semiconductors}
\newacronym{ST}{ST}{STMicroelectronics}
\newacronym{GPIO}{GPIO}{General Purpouse Input/Output}
\newacronym{IRQ}{IRQ}{{\em Interrupt request}, Petició d'interrupció}
\newacronym{ISR}{ISR}{{\em Interrupt Service Routine}, Rutina de servei d'interrupció}
\newacronym{DMA}{DMA}{\em Direct Memory Access}
\newacronym{USART}{USART}{\em Universal Synchronous and Synchronous Receiver-Transmitter}
\newacronym{UART}{UART}{\em Universal Asynchronous Receiver-Transmitter}
\newacronym{bps}{bps}{Bits per segon}
\newacronym{I2C}{I2C}{Inter-Integrated Circuit}
\newacronym{SPI}{SPI}{Serial Peripheral Interface}
\newacronym{MOSI}{MOSI}{{\em Master Output Slave Input}, Sortida del Master Entrada a l'Esclau}
\newacronym{MISO}{MISO}{{\em Master Input Slave Output}, Entrada al Master Sortida de l'Esclau}
\newacronym{SCLK}{SCLK}{{\em Serial Clock}, Relloge Master}
\newacronym{SS}{SS}{{\em Slave Select}, Selecció d'Esclau}
\newacronym{RTC}{RTC}{Real Time Clock}
\newacronym{IDE}{IDE}{{\em Integrated development environment}, Entorn integrat de desenvolupament}
\newacronym{BSP}{BSP}{\em Board Support Package}
\newacronym{Vdd}{Vdd}{Voltatge d'alimentació}
\newacronym{SCL}{SCL}{Serial Clock Line (I2C)}
\newacronym{SDA}{SDA}{Serial Data (I2C)}
\newacronym{USB}{USB}{Universal Serial Bus}
\newacronym{RTOS}{RTOS}{{\em Real-Time Operating System}, Sistema Operatiu de Temps Real}
\newacronym{OS}{OS}{{\em Operating System}, Sistema Operatiu}
\newacronym{pdFALSE}{pdFALSE}{Valor lògic ``Fals'' definit a FreeRTOS}
\newacronym{pdTRUE}{pdTRUE}{Valor lògic ``Cert'' definit a FreeRTOS}
\newacronym{MCU}{MCU}{\em MicroController Unit}
\newacronym{MAC}{MAC}{{\em Multiply and Accumulate} Instrucció de multiplicar i acumular}
\newacronym{SIMD}{SIMD}{{\em Single Instruction, Multiple Data} Única instrucció, Múltiples dades}
\newacronym{RTTI}{RTTI}{{\em Run-time type information} Informació de tipus en temps d'execució}
\newacronym{LCD}{LCD}{{\em Liquid Crystal Display} Pantalla de Cristall Líquid}
\newacronym{MII}{MII}{\em Media-independent interface}
\newacronym{SDIO}{SDIO}{\em Secure Digital Input Output}
\newacronym{CAN}{CAN}{\em Controller area network}
\newacronym{IoT}{IoT}{\em Internet Of Things}
\newacronym{MCI}{MCI}{\em Memory Card Interface}
\newacronym{SAI}{SAI}{\em Serial Audio Interface}
\newacronym{API}{API}{\em Application programming interface}

\newglossaryentry{Cortex}{
name={Cortex},
description={Arquitectura de microcontroladors de la companyia ARM}
}

\newglossaryentry{FW}{
name={FW},
description={{\em Firmware}, Aquell software guardat i executat en un microcontrolador}
}

\newglossaryentry{CPU}{
name={CPU},
description={{\em Central Processor Unit} Unitat central de procès, part que fa tot el còmput i maneig de dades dins un processador}
}

\newglossaryentry{ARM}{
name={ARM},
description={Empresa anglesa que dissenya i comercialitza arquitectures de processadors i eines associades}
}

\newglossaryentry{PCB}{
name={PCB},
description={{\em Printed Circuit Board} Placa de circuit imprès on es solden els components Hardware}
}

\newglossaryentry{FLASH}{
name={FLASH},
description={Memòria no-volàtil de gran capacitat i baix consum}
}

\newglossaryentry{pull-up}{
name={Pull-up},
description={Resistència connectada a alimentació per forçar un valor d''1' a una línia o bus}
}

\newglossaryentry{pull-down}{
name={Pull-down},
description={Resistència connectada a terra per forçar un valor d''0' a una línia o bus}
}

\newglossaryentry{PLC}{
name={PLC},
description={Programmable logic controller, autòmats industrials}
}

\newglossaryentry{Timer}{
name={Timer},
description={Comptador segons un rellotge que genera una interrupció quan arriba a un cert valor configurable}
}

\newglossaryentry{Watchdog}{
 name={Watchdog},
 description={Perifèric que reinicia el sistema si no s'hi accdeix periòdicament}
}

\newglossaryentry{dead-lock}{
name={Dead-lock},
description={Situació en que dues o més tasques estan bloquejades esperant-se una a l'altra}
}

\newglossaryentry{memory mapped}{
name={\em memory mapped},
description={Assignar una posició de memòria a un component HW per facilitar el seu accés},
sort=memory mapped
}

\newglossaryentry{PWM}{
name={PWM},
description={Pulse Width Modulation}
}

\newglossaryentry{callback}{
name={Callback},
description={Funció que es crida en quan acaba un procés}
}

\newglossaryentry{buffer circular}{
name={Buffer circular},
description={Estructura de dades que utilitza un sol buffer i que es pot accedir-ho de forma circular}
}

\newglossaryentry{duty cycle}{
 name={Duty cycle},
 description={Cicle de treball, és la relació que existeix entre el temp que un senyal esta a '1' i el periode d'aquest senyal}
}

\newglossaryentry{CP2102}{
name={CP2102},
description={Dispositiu conversor de USB a sèrie, força utilitzat per comunicar microcontroladors amb ordinadors a través del port USB}
}

\newglossaryentry{FreeRTOS}{
name={FreeRTOS},
description={Sistema Operatiu de Temps Real de codi obert i lliure}
}

\newglossaryentry{task}{
name={Task},
description={Funció que implementa un procés de SO juntament amb l'{\em stack} corresponent}
}

\newglossaryentry{tick}{
name={\em Tick},
description={Esdeveniment periòdic per a que el Sistema Operatiu prengui el controll de l'execució},
sort=Tick
}

\newglossaryentry{stack}{
name={\em Stack},
description={Regió de memòria on emmagatzemar dades pròpies d'una tasca},
sort=Stack
}

\newglossaryentry{race condition}{
name={\em race condition},
description={Error provocat per l'exeucuó simultànea de dues o més tasques},
sort=Race condition
}

\newglossaryentry{flag}{
name={\em flag},
description={Un {\em flag} és un bit que indica algun valor determinat d'un dispositiu o dada},
sort=Flag
}

\newglossaryentry{open drain}{
name={\em open drain},
description={Tipus de sortida que només pot forçar un valor de '0', per tant li cal un \gls{pull-up} per funcionar correctament},
sort=Open
}

\newglossaryentry{CMSIS}{
name={CMSIS},
description={\em Cortex Microcontroller Software Interface Standard},
sort=CMSIS
}

\newglossaryentry{GCC}{
name={{GCC}},
description={\em GNU C Compiler], Compilador de C de GNU},
sort=GCC
}

\newglossaryentry{Systick}{
name={\em Systick},
description={{\em Timer} integrat dins els {\em cores} Cortex-M},
sort=Systick
}

\newglossaryentry{layout}{
name={\em layout},
description={Dibuix final de com queda la PCB a dissenyar},
sort=layout
}

\newglossaryentry{macro}{
name={\em macro},
description={Tros de codi que té assignat un nom, de manera que cada cop que s'inserta el nom es canvia pel codi definit prèviament},
sort=macro,
}
